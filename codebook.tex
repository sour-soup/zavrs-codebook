%!TEX TS-program = xelatex
\documentclass[a4paper,10pt,twocolumn,oneside]{article}
\setlength{\columnsep}{10pt}  
\setlength{\columnseprule}{0pt}      

\usepackage{amsthm}	
\usepackage{amssymb}
%\usepackage[margin=2cm]{geometry}
\usepackage{fontspec}
\usepackage{color}
\usepackage[x11names]{xcolor}
\usepackage{listings}	
\usepackage[Glenn]{fncychap}
\usepackage{fancyhdr}
\usepackage{graphicx}
\usepackage[export]{adjustbox}								%Graphic
\usepackage{enumerate}
\usepackage{changepage}
\usepackage{titlesec}
\usepackage{amsmath}
\usepackage{codebook}
\usepackage{verbatim}
\usepackage[CheckSingle, CJKmath]{xeCJK}
% \usepackage{CJKulem}

%\usepackage[T1]{fontenc}
\usepackage{amsmath, courier, listings, fancyhdr, graphicx}
\topmargin=0pt
\headsep=5pt
\textheight=780pt
\footskip=0pt
\voffset=-40pt
\textwidth=545pt
\marginparsep=0pt
\marginparwidth=0pt
\marginparpush=0pt
\oddsidemargin=0pt
\evensidemargin=0pt
\hoffset=-42pt

%%%%%%%%%%%%%%%%%%%%%%%%%%%%%

\setmainfont{Consolas}	
\setmonofont{Monaco}
%\setCJKmainfont{Source Han Sans}
% \setCJKmainfont{Consolas}
%\setmainfont{sourcecodepro}
\XeTeXlinebreaklocale "zh"
\XeTeXlinebreakskip = 0pt plus 1pt
\setcounter{secnumdepth}{3}

%%%%%%%%%%%%%%%%%%%%%%%%%%%%%
\makeatletter
\lst@CCPutMacro\lst@ProcessOther {"2D}{\lst@ttfamily{-{}}{-{}}}
\@empty\z@\@empty
\makeatother
\lstset{											% Code
language=C++,										% the language of the code
basicstyle=\ttfamily, 						% the size of the fonts that are used for the code
%numbers=left,										% where to put the line-numbers
numberstyle=\footnotesize,						% the size of the fonts that are used for the line-numbers
stepnumber=1,										% the step between two line-numbers. If it's 1, each line  will be numbered
numbersep=5pt,										% how far the line-numbers are from the code
backgroundcolor=\color{white},					% choose the background color. You must add \usepackage{color}
showspaces=false,									% show spaces adding particular underscores
showstringspaces=false,							% underline spaces within strings
showtabs=false,									% show tabs within strings adding particular underscores
frame=false,											% adds a frame around the code
tabsize=2,											% sets default tabsize to 2 spaces
captionpos=b,										% sets the caption-position to bottom
breaklines=true,									% sets automatic line breaking
breakatwhitespace=false,							% sets if automatic breaks should only happen at whitespace
escapeinside={\%*}{*)},							% if you want to add a comment within your code
morekeywords={*},									% if you want to add more keywords to the set
keywordstyle=\bfseries\color{Blue1},
commentstyle=\itshape\color{Red4},
stringstyle=\itshape\color{Green4},
literate={\ \ }{{\ }}1
}

\university{1}{Saratov State University}{SSU.png}
\team{ZAVRS}{Eremenko, Utaliev, Yanchenko}
\contest{Contest name}{Contest Date}

%%%%%%%%%%%%%%%%%%%%%%%%%%%%%

\begin{document}
\maketeampage

\pagestyle{fancy}
\fancyfoot{}
%\fancyfoot[R]{\includegraphics[width=12pt]{SSU.png}}
\fancyhead[L]{Saratov State University: Eremenko, Utaliev, Yanchenko}

\fancyhead[R]{\thepage}
\renewcommand{\headrulewidth}{0.4pt}
\renewcommand{\contentsname}{Contents} 
\scriptsize
\begingroup
\let\clearpage\relax
% \tableofcontents
\endgroup
\tableofcontents
%%%%%%%%%%%%%%%%%%%%%%%%%%%%%
% \newpage


\section{Basic}
\code{Pragma optimization}{basic/pragma.cpp}
\code{Hash Function}{basic/hash.cpp}

\section{Math}
\code{Integration}{math/integration.cpp}
\code{Simulated Annealing}{math/annealing.cpp} %удалим, если что
\code{Gauss}{math/gauss.cpp}
\code{Iteration}{math/iteration.cpp} %удалим, если что
\codewithtext{Tridiagonal Matrix Algorithm}{math/tridiag.cpp}{$a_{i}x_{i-1}+b_{i}x_{i}+c_{i}x_{i+1}=d_{i}$} %удалим, если что


\section{Number Theory}
\code{FFT}{nt/fft.cpp} %поменять
\code{NTT}{nt/ntt.cpp} %поменять
\code{FWHT}{nt/fwht.cpp}
%поправить
\code{Extended Euclid}{nt/exgcd.cpp}
\tex{Burnside Lemma}{nt/burnside.tex}
\code{Phi and Mobius}{nt/phimu.cpp}
\code{Pollard Rho}{nt/pollard.cpp} %написать че делает и ассимптотику
\code{Miller Rabin}{nt/miller.cpp} %написать че делает и ассимптотику
\code{Primitive Root}{nt/primitive.cpp} %пусть болтается
\code{Discrete Root}{nt/discrete\_root.cpp} %пусть болтается
%добавить дискретный логарифм
\codewithtext{Diophantine Equations}{nt/diophantine.cpp}{solutions to $ax + by = c$ where $x \in [xlow, xhigh]$ and $y \in [ylow, yhigh]$; returns \{cnt, leftsol, rightsol, gcd of a and b\}}
\code{Stern-Brocot Tree}{nt/stern-brocot.cpp} %можно переписать
\codewithtext{Lattice Points Below Line}{nt/lattice\_points.cpp}{Number of integer points $(x;y)$ such for $0 \leq x < n$ and $0 < y \leq \lfloor k x+b\rfloor$}
%добавить структуру для рациональных чисел


\section{Graph Theory}
\code{Biconnected Components}{graph/bicon.cpp} %переписать
\code{2-SAT}{graph/twosat.cpp}
\tex{Counting Labeled Graphs}{graph/labeled\_graphs.tex} %Вадим бы это перевёл
\tex{Kirchoffs Theorem}{graph/kirghoff.tex}
\tex{Tuttes Theorem}{graph/tutte.tex} %может вадим перепишет
\tex{Matching Duals}{graph/matching\_dual.tex}
%добавить lca
%добавить куна
%добавить форда беллмана, флойда
%добавить борувку
%добавить эйлеров путь


\section{Flows}
\code{Dinics Algorithm}{flows/dinics.cpp}
\code{MCMF}{flows/mcmf.cpp}
%следующее оставляем как черный ящик, если что, удалим
\codewithtext{MCMF with Potentials}{flows/mcmf\_potentials.cpp}{call init\_dag() or init\_fb() (depending on if your graph is dag or not) before running calc()} 
\tex{L-R Flow}{flows/lrflow.tex}
\code{Stoer Wagner Algorithm}{flows/stoer-wagner.cpp} %нужно написать разъяснения, ограничение и время
\code{Hungarian Algorithm}{flows/hungarian.cpp} %нужно написать че делает
\code{Blossom Algorithm}{flows/blossom.cpp} %нужно написать че делает
%нужно добавить задачу про проекты и инструменты


\section{Data Structures}
\code{Centroid Decomposition}{ds/centroid.cpp} %нужно разобраться
\code{HLD}{ds/hld.cpp}
\code{Explicit Treap}{ds/treap.cpp} %можно переписать
%добавить sparse table
%добавить персистентное до
%добавить персистентный декартач
\code{Ordered Set}{ds/pbds.cpp}


\section{Strings}
\tex{String Matching with FFT}{strings/fft\_match.tex}
\codewithtext{Suffix Array}{strings/suffix\_array.cpp}{Don't forget to modify the string to avoid cyclic comparisons if needed}
\code{Prefix- and z-funtion}{strings/prefix-z.cpp}
\code{Manachers Algorithm}{strings/manacher.cpp}
\code{Prefix Automaton}{strings/prefix\_automaton.cpp}
\code{Aho-Corasick + Trie}{strings/aho.cpp}
\code{Hash String}{strings/hash.cpp}


\section{DP}
\code{Dynamic CHT}{dp/cht.cpp} %пояснить че делает
\code{Li Chao Tree}{dp/lichao.cpp} %надо бы разобраться
\code{D\&C Optimization}{dp/dnc.cpp} %надо написать условия
%нужно переписать условие
\codewithtext{Knuth Optimization}{dp/knuth.cpp}{Достаточное условие: 1. $C_{ac}+C_{bd}\le C_{ad}+C_{bc}$, 2. $C_{bc}\le C_{ad}$, при всех $a\le b\le c\le d$.}
%это надо переписать
\codewithtext{Slope Trick}{dp/slope.cpp}{You are given an array, each operation you are allowed to increase or decrease an element's value by $1$. Find the minimum number of operations to make the array strictly increasing.}
%добавить лямбда оптимизацию
%добавить сос дп
%добавить проход по подмаскам
%добавить нвп бинпоиском


\section{Geometry}
\code{Base}{geometry/base.cpp}
\code{Intersections}{geometry/intersections.cpp}
\code{Angles}{geometry/angles.cpp}
\code{Areas}{geometry/areas.cpp}
\code{Distances}{geometry/distances.cpp}
%добавить выпуклую оболочку
\code{Circle Circle Area}{geometry/circle\_circle\_area.cpp}
\code{Circle Line Intersection}{geometry/circle\_line.cpp}
\code{Circle Circle Intersection}{geometry/circle\_circle.cpp}
\code{Tangent Lines of Two Circles}{geometry/tangents\_circles.cpp}
\codewithtext{Minkowski Sum}{geometry/minkowski.cpp}{Рассмотрим два множества \( A \) и \( B \) точек на плоскости. Сумма Минковского \( A + B \) определяется как \( \{a + b \mid a \in A, b \in B\} \). Будем рассматривать случай, когда \( A \) и \( B \) состоят из выпуклых многоугольников \( P \) и \( Q \) с их внутренностями. Cумма выпуклых многоугольников \( P \) и \( Q \) является выпуклым многоугольником с не более чем \( |P| + |Q| \) вершинами.}
\code{Point in Convex Polygon}{geometry/point\_in\_convex.cpp}
\code{SVG}{geometry/svg.cpp}


\section{Miscellaneous}
%написать че делает
\code{Josephus Problem}{misc/josephus.cpp} 
%надо написать че делает
\code{Knight Moves in Infinity Grid}{misc/knight.cpp}

\setlength{\parindent}{0cm}


\section{Other}
%можно добавить факты про псп
\tex{Комбинаторика}{other/combinatorics.tex}
%нужно добавить сумму делителей
\tex{Теория чисел}{other/number-theory.tex}
%нужно исправить
%нужно добавить формулы для многоугольников
\tex{Геометрия}{other/geometry.tex}
\tex{Графы}{other/graphs.tex}
%нужно переписать интерполяционный многочлен ньютона
\tex{Формулы}{other/other.tex}
\tex{Полезные числа}{other/table.tex}

\end{document}