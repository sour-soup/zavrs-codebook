Нам даны две строки, текст $T$ и шаблон $P$, состоящие из строчных букв. Мы должны найти все вхождения шаблона в текст.
Шаблон может содержать символ джокера *, который соответствует любому символу.

1. Преобразуйте текст $T$ и шаблон $P$ в полиномы:
   Для строки текста:
   \[ A(x) = a_0 x^0 + a_1 x^1 + \dots + a_{n-1} x^{n-1} \]
   где \( a_i = \cos(\alpha_i) + i \sin(\alpha_i), \quad \alpha_i = \frac{2 \pi T[i]}{26} \)
   
   Для шаблона:
   \[ B(x) = b_0 x^0 + b_1 x^1 + \dots + b_{m-1} x^{m-1} \]
   где \( b_i = \cos(\beta_i) - i \sin(\beta_i), \quad \beta_i = \frac{2 \pi P[m-i-1]}{26} \)
   
   (шаблон разворачивается для сопоставления).

2. Умножьте полиномы с помощью FFT, получив произведение:
   \[ C(x) = A(x) \cdot B(x) \]
   Коэффициент \( c_{m-1+i} \) укажет, есть ли совпадение шаблона на позиции \( i \).
3. Если в шаблоне есть джокеры (символы $*$), установите для них соответствующие коэффициенты полинома \( b_i = 0 \).

4. Совпадение шаблона с текстом на позиции \( i \) происходит, если:
   \[ c_{m-1+i} = m - x \]
   где \( x \) — количество джокеров в шаблоне.

Таким образом, вы сможете искать строки и шаблоны с джокерами через умножение полиномов с FFT.