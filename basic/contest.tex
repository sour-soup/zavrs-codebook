\textbf{Перед отправкой:}  

- Напишите несколько простых тестовых примеров, если сэмплов недостаточно.  

- Ограничение по времени близко к пределу? Если да, сгенерируйте макстесты.  

- Используется ли память в допустимых пределах?  

- Возможны ли переполнения?  

- Убедитесь, что отправляете правильный файл.  


\textbf{Неверный ответ:}  

- Напечатайте ваше решение! Также выведите отладочную информацию.  

- Очищаются ли все структуры данных между тестами?  

- Может ли алгоритм обработать весь диапазон входных данных?  

- Прочитайте условие задачи еще раз.  

- Учтены ли все граничные случаи?  

- Поняли ли вы задачу правильно?  

- Есть ли неинициализированные переменные?  

- Возможны ли переполнения?  

- Не перепутаны ли похожие переменные, например, $N$ и $M$, $i$ и $j$?  

- Уверены ли вы, что ваш алгоритм работает корректно?  

- Какие особые случаи вы могли упустить?  

- Уверены ли вы, что используемые функции STL работают так, как вы предполагаете?  

- Пройдитесь по алгоритму на простом примере.  

- Пройдитесь по этому списку еще раз.  

- Объясните алгоритм коллеге.  

- Попросите коллегу взглянуть на ваш код.  

- Сделайте небольшую паузу (например, прогуляйтесь).  

- Формат вывода правильный? (включая пробелы)  

- Попробуйте переписать решение с нуля или доверьте это коллеге.  


\textbf{Ошибка времени выполнения:}  

- Проверили ли вы все граничные случаи локально?  

- Есть ли неинициализированные переменные?  

- Читаете или записываете ли вы за пределами массива?  

- Возможны ли деления на 0? (например, $mod 0$)  

- Возможна ли бесконечная рекурсия?  

- Есть ли невалидные указатели или итераторы?  

- Используете ли слишком много памяти?  


\textbf{Превышение лимита времени:}  

- Есть ли в коде возможные бесконечные циклы?  

- Какова сложность вашего алгоритма?  

- Копируете ли вы много ненужных данных? (используйте ссылки)  

- Каков объем входных и выходных данных?

- Что коллеги думают о вашем алгоритме?  


\textbf{Превышение лимита памяти:}  

- Сколько памяти максимально потребуется вашему алгоритму?  

- Очищаете ли вы все структуры данных между тестами?  