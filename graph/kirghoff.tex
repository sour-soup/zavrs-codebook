Возьмём матрицу смежности графа $G$, заменим каждый элемент этой матрицы на противоположный, а на диагонали вместо элемента $A_{i,i}$ поставим степень вершины $i$ (если имеются кратные рёбра, то в степени вершины они учитываются со своей кратностью). Тогда, согласно матричной теореме Кирхгофа, все алгебраические дополнения этой матрицы равны между собой, и равны количеству остовных деревьев этого графа. Например, можно удалить последнюю строку и последний столбец этой матрицы, и модуль её определителя будет равен искомому количеству.