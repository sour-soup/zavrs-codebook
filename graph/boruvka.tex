Алгоритм Борувки опирается на этот факт и заключается в следующем:
\begin{enumerate}
    \item Для каждой вершины найдем минимальное инцидентное ей ребро.
    \item Добавим все такие рёбра в остов и сожмем получившиеся компоненты, то есть объединим списки смежности вершин, которые эти рёбра соединяют.
    \item Повторяем шаги 1-2, пока в графе не останется только одна вершина-компонента.
\end{enumerate}

Алгоритм может работать неправильно, если в графе есть ребра, равные по весу. Пример: «треугольник» с одинаковыми весами рёбер. Избежать такую ситуацию можно, введя какой-то дополнительный порядок на рёбрах — например, сравнивая пары из веса и номера ребра.

Заметим, что на каждой итерации каждая оставшаяся вершина будет задействована в «мердже». Это значит, что количество вершин-компонент уменьшится хотя бы вдвое, а значит всего итераций будет не более $O(log{n})$.
Итоговая сложность: $O(m log{n})$

Алгоритм полезен на неявных графах, если мы можем быстро находить минимальное ребро вершины.