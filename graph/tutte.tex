Матрицей Татта называется следующая матрица $n \times n$: $A_{ij} (1 \le i < j \le n)$~--- это либо независимая переменная, соответствующая ребру между вершинами $i$ и $j$, либо тождественный ноль, если ребра между этими вершинами нет. $A_{ji} = -A_{ij}$ .

В графе $G$ существует совершенное паросочетание тогда и только тогда, когда $\det(A)$. На практике подставляем случайные числа, делаем несколько итераций.