\textbf{Разбиение числа $n$ на $k$ слагаемых:}

$C_{n + k - 1}^{k - 1}=C_{n + k - 1}^n$ --- для $a_i \ge 0$,
$C_{n - 1}^{k - 1}$ --- для $a_i \ge 1$,
$C_{n - k - 1}^{k - 1}$ --- для $a_i \ge 2$.

\textbf{Биномиальные коэффициенты:}

Способы подсчёта $\binom{n}{k}=\frac nk \binom{n-1}{k-1}$:

\begin{enumerate}

\item $\binom{n}{k}$ можно получить из $\binom{n}{0}$ за $O(k)$, с помощью формулы $\binom{n}{k}=\frac{n-k+1}k \binom{n}{k-1}$.

\item По простому (маленькому) модулю можно считать с помощью теоремы Люка:

$\displaystyle \binom{m}{n} \equiv \prod\limits_{i = 0}^{k - 1} \binom{m_i}{n_i}\ (mod\ p),\ p$ --- простое.
$\displaystyle m = (m_{k - 1}, \dots, m_0)_p, n = (n_{k - 1}, \dots, n_0)_p$ --- представление чисел $m$ и $n$ в $p$-ичной системе счисления.

\item Предподсчитать факториалы за $O(n+m)$. При $n, k \sim 10^9$ можно предподсчитать в коде каждый $10^6$-й факториал.

\item Можно для каждого простого найти степень, с которой оно входит в $\binom{n}{k}$ и далее получить само число.

\item Если $n\sim 10^{18}$, а $k\sim 10^6$ --- маленькое, то $\binom{n}{k}$ по простому модулю (в случае составного надо использовать КТО)
можно искать с помощью факториального представления: рекурсивная функция считает произведение по всем некратным $p$, по кратным вычисляет
степень $p$ и далее рекурсивно вызывает себя (поскольку кратные без коэффициента $p$ образуют новые факториалы).

\end{enumerate}

$\overline{C}_n^k=C_{n + k - 1}^k$ --- число сочетаний с повторениями из $n$ по $k$,

$A_n^k=\frac{n!}{(n-k)!}=C_n^k\:k!$ --- количество размещений из $n$ по $k$,

$\sum\limits_{k=0}^{n}k C_n^k=n2^{n-1}$, $\sum\limits_{k=0}^{n}k^2 C_n^k=(n+n^2)2^{n-2}$,
$\sum\limits_{k=0}^{n}{(C_n^{k})^2}=C_{2n}^n$,

$\sum\limits_{k=0}^{\lfloor\frac{n}{2}\rfloor}C_{n-k}^k=F(n+1)$, где $F(n)$ --- $n$-ое число Фибоначчи,

$\sum\limits_{k\le n} \binom{r+k}{k}=\binom{r+n+1}{n}, n\in \mathbb{Z}$ --- параллельное суммирование,

$\sum\limits_{0\le k\le n} \binom{k}{m} = \binom{n+1}{m+1}, m, n\in \mathbb{Z}, m, n\ge 0$ --- верхнее суммирование,

$\sum\limits_k \binom{r}{k}\binom{s}{n-k}=\binom{r+s}{n}, n\in \mathbb{Z}$ --- свёртка Вандермонда.

\textbf{Числа Каталана:} $\displaystyle C_n = \frac1{n + 1}\binom{2n}{n}$.

\textbf{Числа Стирлинга:}

1-го рода: $Z_n^k =(n-1)Z_{n - 1}^k + Z_{n-1}^{k-1}$ --- число способов разбиения множества из $n$ элементов на $k$ циклов (циклы нельзя переворачивать, только прокручивать).

2-го рода: $S_n^k = kS_{n - 1}^k + S_{n-1}^{k-1} = \frac1{k!} \sum\limits_{j=0}^k (-1)^{k+j}\binom{k}{j}j^n$ --- число способов разбиения множества из $n$ элементов на $k$ непустых подмножеств.

Полагаем, что $Z_n^0=S_n^0=[n=0], Z_0^k=S_0^k=[k=0]$. Дуальность: $S_n^k=Z_{-k}^{-n}$.

$x^n=\sum\limits_k S_n^k x^{\underline{k}}$,
$x^{\overline{n}}=\sum\limits_k Z_n^k x^k$,
$x^n=\sum\limits_k S_n^k (-1)^{n-k} x^{\overline{k}}$,
$x^{\underline{n}}=\sum\limits_k Z_n^k (-1)^{n-k} x^k, n\in \mathbb{Z}, n \ge 0$.

\textbf{Количество неубывающих последовательностей из $n$ элементов от $0$ до $a$ равно $\binom{n + a}{n}$.}

\textbf{Перестановки без неподвижных точек:} $cnt = n! \sum\limits_{i = 0}^n \frac{(-1)^n}{i!} \approx \frac{n!}e$

\textbf{Число Непера:} $e = \sum\limits_{n = 0}^\infty \frac1{n!}, \frac1e = \sum\limits_{n = 2}^\infty \frac{(-1)^n}{n!}$

\textbf{Количество диаграмм Юнга}

Крюк клетки --- она сама, а также клетки, расположенные справа от нее, и клетки, расположенные снизу.

Количество заполнений таблицы равно факториалу количества ее клеток, деленному на произведение длин всех крюков.

\textbf{Теорема Каммера:}

$n$, $m\in Z$, $n\geq  m\geq  0$ и $p$ --- простое число, тогда максимальная степень $k$, что $p^k | C_n^m$ равна количеству переносов при сложении чисел $m$ и $n - m$ в системе счисления $p$.

