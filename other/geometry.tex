\textbf{Теорема синусов:} $\frac{a}{\sin\alpha}=\frac{b}{\sin\beta}=\frac{c}{\sin\gamma}=2R,\ R$ --- радиус описанной окружности.

\textbf{Теорема косинусов:} $\displaystyle a^2 = b^2 + c^2 - 2bc\sin{\widehat{bc}}$.

\textbf{Теорема тангенсов:} $\displaystyle \frac{a-b}{a+b}=\frac{\tan\frac{1}{2}(\alpha -\beta)}{\tan\frac{1}{2}(\alpha +\beta)}$.

\textbf{Геометрия на сфере:}

1. Большой круг задаётся нормалью.

2. Два неравных круга пересекаются в двух противоположных точках: $\pm normal(cross(n_1, n_2), r)$.

3. Площадь многоугольника равна: $S = (\sum\limits_i\alpha_i - \pi * (n - 2)) * r^2$,
объем конуса (сектора) равен: $V = S * r / 3$ ($\alpha_i$ --- внутренний двугранный угол = угол между касательными векторами).

4. $\alpha_i = \pi - ang(n_i, n_{i + 1})$, где $n_j$ --- нормаль к $j$-й дуге многоугольника направленная внутрь многоугольника.

5. Для сортировки точек на круге $C$ удобно перейти к $2D$ (введя базис в плоскости круга:
$e_1 = p_0, e_2 = cross(e_1, n_c)), p_0$ --- любая из точек на круге, $n_c$ --- нормаль к кругу.

6. Для сортировки по кругу отрезков исходящих из точки $C$, надо свести задачу к 5.

\textbf{Треугольники и их свойства:}
$a, b, c$ --- стороны, треугольника;
$\alpha, \beta, \gamma$ --- соответствующие сторонам $a, b, c$ углы,
$r, R$ --- радиусы вписанной и описанной окружностей,
$d$ --- расстояние между центрами вписанной и описанной окружностей.

\begin{itemize}
    \item[$1.$] $S=\frac{1}{2}ab\sin\gamma=\frac{a^2\sin \beta\sin \gamma}{2\sin\alpha}=2R^2\sin\alpha\sin\beta\sin\gamma$,
	\item[$2.$] $S=\frac{ab}{2}=r^2+2rR, r = \frac{ab}{a + b + c} = \frac{a + b - c}2$ --- для прямоугольного,
	\item[$3.$] $S=\frac{a^2\sqrt 3}{4}$ --- для равностороннего,
    \item[$4.$] $R=\frac{abc}{4S}$ --- центр в точке пересечения серединных перпендикуляров,
	$r=\frac Sp$ --- центр в точке пересечения биссектрис, $d^2=R^2-2Rr$.
\end{itemize}

\textbf{Геометрическая инверсия}

Инверсия точки $P$ относительно окружности с центром в точке $O$ и радиусом $R$ --- это точка $P'$, которая лежит на луче $OP$, и $OP \cdot OP' = R^2$.

Прямая, проходящая через $O$, не меняется. Прямая, не проходящая через $O$, перейдет в окружность, проходящую через $O$, и наоборот. Окружность, не проходящая через $O$, перейдет в окружность, по-прежнему не проходящую через $O$.

Если после инверсии точки $P$ и $Q$ переходят в $P'$ и $Q'$, то $\angle PQO = \angle Q'P'O$, $\angle QPO = \angle P'Q'O$ и треугольники $\triangle PQO$ и $\triangle Q'P'O$ подобны.

Преобразование инверсии сохраняет углы в точках пересечения кривых (ориентация меняется на противоположную).

Обобщённая окружность при преобразовании инверсии сохраняется тогда и только тогда, когда она ортогональна окружности, относительно которой производится инверсия.

Чтобы найти окружность, получившуюся в результате инверсии прямой, нужно найти ближайшую к центру инверсии точку $Q$ прямой, применить к ней инверсию, и тогда искомая окружность будет иметь диаметр $OQ'$.

Чтобы найти окружность, получившуюся в результате инверсии другой окружности, нужно провести через центр инверсии и центр старой окружности прямую, и посмотреть ее точки пересечения $S$ и $T$ со старой окружностью. Отрезок $ST$ после инверсии будет образовывать диаметр, следовательно, центр новой окружности это среднее арифметическое точек $S'$ и $T'$.

Окружность с центром в точке $(x, y)$ и радиусом $r$ после инверсии относительно окружности с центром в точке $(x_0, y_0)$ и радиусом $r_0$ перейдет в окружность с центром в точке $(x', y')$ и радиусом $r'$, где

$x' = x_0 + s \cdot (x - x_0)$,
$y' = y_0 + s \cdot (y - y_0)$,
$r' = |s| \cdot r$,
$s = \frac { r_0^2 } { (x - x_0)^2 + (y - y_0)^2 - r^2 }$.

\textbf{Шар:}

$S=4\pi R^{2},\; V=\frac{4}{3}\pi R^{3}$
$V=V_{n}R^{n},\; V_{n}=\frac{\pi ^{\lfloor n/2 \rfloor}}{\Gamma (\frac{n}{2} + 1)},\; V_{2k}=\frac{\pi^{k}}{k!},\; V_{2k+1}=\frac{2^{k+1}\pi^{k}}{(2k+1)!!}$

$S=S_{n}R^{n},\; S_{0}=2,\; S_{n}=2\pi V_{n - 1}$

$V=\frac{1}{3}\pi h^{2}(3R-h),\; S=2\pi Rh$ для шарового сегмента, $h$ --- высота сегмента

$V=\frac{2}{3}\pi R^{2}h$ для шарового сектора, $h$ --- высота соответствующего шарового сегмента

\textbf{Тор:}

$S=4\pi^2Rr,\; V=2\pi^2Rr^2$, $R$ --- расстояние от центра образующей окружности до оси вращения, $r$ --- радиус образующей окружности

\textbf{Тетраэдр:}

$V=\frac{\sqrt 2 a^3}{12},\; h=\frac{\sqrt 6 a}{3},\; r=\frac{\sqrt 6 a}{12},\; R=\frac{\sqrt 6 a}{4}$,

$288V^2=
\begin{vmatrix}
0 & 1 & 1 & 1 & 1\\ 
1 & 0 & d_{12}^{2} & d_{13}^{2} & d_{14}^{2}\\ 
1 & d_{12}^{2} & 0 & d_{23}^{2} & d_{24}^{2}\\ 
1 & d_{13}^{2} & d_{23}^{2} & 0 & d_{34}^{2}\\ 
1 & d_{14}^{2} & d_{24}^{2} & d_{34}^{2} & 0
\end{vmatrix}$. При фиксированных попарных расстояниях тетраэдр построить нельзя, если определитель $<0$ или хотя бы на одной грани не выполняется неравенство треугольника.

\textbf{Конус:}

$V=\frac{1}{3}\pi R^2 h$,
$S=\pi Rl$ --- площадь боковой поверхности, где $l$ --- образующая.

\textbf{Круг:}

$S=\frac{R^2}{2}\theta$ для сектора круга,
$S=\frac{R^2}{2}(\theta - \sin\theta)$ для сегмента круга.

\textbf{Формула Эйлера для числа граней в планарном графе:}

$V-E+F=1+C$, где $V$ - число вершин, $E$ - ребер, $F$ - граней, $C$ - компонент связности графа.

\textbf{Теорема о секущих:}

Если из точки, лежащей вне окружности, провести две секущие, то произведение одной секущей на её внешнюю часть равно произведению другой секущей на её внешнюю часть:
$AB\cdot AC=AD\cdot AE$

\textbf{Пересечение плоскости и прямой:}

Дана плоскость $Ax+By+Cz+D=0$ и прямая в виде $a+vt$, тогда точке пересечения будет соответствовать параметр

$t=-\frac{A \cdot a.x+B \cdot a.y+C \cdot a.z+D}{A \cdot v.x+B \cdot v.y+C \cdot v.z}$

\textbf{Сферические координаты:}

Если $\theta\in \left [ -\frac{\pi}{2}, \frac{\pi}{2} \right ]$ --- широта, а
$\varphi\in \left [ 0, 2\pi \right )$ --- долгота, то

$x=r\cos\theta \cos\varphi$,
$y=r\cos\theta \sin\varphi$,
$z=r\sin\theta$.

\textbf{Пересечение окружности и прямой:}
Сдвинем систему координат в центр окружности. Для сдвига прямой на вектор $(dx, dy)$ нужно сделать $C -= A \cdot dx + B \cdot dy$,
в данном случае $dx = -x_0, dy = -y_0$. Теперь если $d > r\ (d=\frac{|C|}{\sqrt{A^2+B^2}})$, то точек пересечения нет.
Обозначим $l=\sqrt{r^2-d^2}, c = (\frac{-AC}{A^2+B^2},\ \frac{-BC}{A^2+B^2})$. Тогда точки пересечения имеют вид:
$c \pm normal(pt(-B, A), l)$.

\textbf{Центры масс:}
Везде далее будем выражать центры масс с помощью радиус-векторов.

Центр масс системы материальных точек: $\vec{r_c}=\frac{\sum \limits_{i} \vec{r_i}\ m_i}{\sum \limits_{i} m_i}$,

Центр масс однородного каркаса, как многоугольника, так и многогранника (заменяем каждое ребро точкой в его середине с массой, равной длине ребра):
$\vec{r_c}=\frac{\sum \limits_{i} \vec{r_{i}^{mid}}\ l_i}{P}$,

Центр масс сплошных фигур: центр масс произвольного сплошного треугольника или тетраэдра - среднее арифметическое его координат (обобщается и на симплексы больших размерностей).
Центр масс произвольного сплошного многоугольника/многогранника считается следующим образом: выбирается произвольная точка $p$
из нее проводятся треугольники/тетраэдры к последовательным вершинам фигуры (или к треугольникам из триангуляции грани)
и вычисляется взвешенное среднее центров масс треугольников/тетраэдров с весами, равными знаковым площадям/объемам.

Центр масс поверхности многогранника - это взвешенное среднее центров масс граней с весами, равными площадям граней.

\textbf{Окружности Мальфатти:}

$r_1=\frac{r}{2(p-a)}(p+d-r-e-f)$, $r_2=\frac{r}{2(p-b)}(p+e-r-d-f)$, $r_3=\frac{r}{2(p-c)}(p+f-r-d-e)$, где $d$, $e$, $f$ --- расстояния от инцентра до углов $A$, $B$, $C$ соответственно

