\textbf{Теорема Холла:} Если в двудольном графе произвольно выбранное множество вершин первой доли
покрывает не меньшее по размеру множество вершин второй доли, то в графе существует совершенный парсоч.

\textbf{Аксиомы} частично упорядоченных множеств (ЧУМ):
$$1.\ a < b \wedge b < c \Rightarrow a < c;\ 2.\ a < b \Rightarrow a \ne b;\ 3.\ a < b \Rightarrow \overline{b < a}.$$

\textbf{Цепь} --- множество попарносравнимых элементов.

\textbf{Антицепь} --- множество попарнонесравнимых элементов.

\textbf{Теорема Дилуорса:} Размер максимальной антицепи равен размеру минимального покрытия ЧУМа цепями.

\textbf{Следствие:} Наибольшее количество вершин в орграфе таких, что никакая вершина недостижима из другой равно
минимальному покрытию замыкания этого графа путями (циклы не мешают).

\textbf{DAG minimum covering:} Необходимо покрыть DAG наименьшим количеством вер\-шин\-но-не\-пе\-ре\-се\-каю\-щих\-ся путей.
Для этого построим двудольный граф вершины есть которого раздвоенные вершины исходного графа.
Дугу $(x, y)$ исходного графа заменяем на дугу $(x_1, y_2)$ нового. Находим макс. парсоч.
Теперь для восстановления ответа достаточно сказать, что вершины соединенные ребром парсоча являются соседними
в одном из путей ответа.

\textbf{Дейкстра с потенциалами:}
Введем потенциалы $\varphi_i$, новые веса ребер будут иметь вид:
$\overline{c_{ij}} = c_{ij} + (\varphi_i - \varphi_j) \ge 0$.
В качестве потенциалов можно выбрать $\varphi_i = d_i$, где $d_i$ --- кратчайшие расстояния из истока
(специально добавленной в граф вершины из которой есть дуги веса $0$ во все остальные вершины)
вычисленные с помощью алгоритма Форда-Беллмана. В частности в задаче $minCostFlow$ можно
положить исходно $\varphi_i = 0$ и после каждого шага увеличивать все $\varphi_i$ на величину $d_i$.
$\varphi_i $ могут переполниться у недостижимых вершин.

\textbf{Паросочетания:}

В произвольном графе: $|MIVS| + |MVC| = |V|, |MM| + |MEC| = |V|$.

В двудольном графе: $|MVC| = |MM|$.

Для нахождения $MVC$ в двудольном графе нужно запустить Куна из ненасыщенных вершин первой доли,
тогда вершины по которым мы не прошли в первой доле и прошли во второй образуют $MVC$ ($MIVS = V \backslash MVC$).

Для нахождения $MEC$ построим $MM$. Теперь из каждой непокрытой (неизолированной) вершины все ребра ведут
в насыщенные вершины. Выберем для каждой ненасыщенной вершины любое ребро и добавим эти ребра в $MM$ получим $MEC$.

$MEC$ представляет собой лес (с деревьями диаметра не более $2$), поэтому для получения $MM$ нужно из каждого дерева
взять по одному ребру.

\textbf{Наибольшее доминирование:} $X$ --- множество вершин первой доли, $Y$ --- множество вершин второй доли которые покрыты вершинами из $X$.
Необходимо найти $X$, чтобы величина $|X| - |Y|$ была максимальна. Для этого пострим $MM$ и запустим Куна из ненасыщенных
вершин первой доли, тогда вершины по которым мы прошли в первой доле образуют множество $X$, а вершины по которым мы прошли во второй
доле образуют множество $Y$ и при этом значение $|X| - |Y|$ будет максимально.

\textbf{2-SAT:}

1. Задачу всегда НЕОБХОДИМО сводить к каноническому виду $2-CNF$: $(a_i \vee b_i) \wedge (a_{i + 1} \vee b_{i + 1})$.
Все бинарные операции легко выражаются в этой форме. В частности импликация $x \rightarrow y$ выражается,
как $(!x \vee y)$.

2. Если заранее известно чему равно значение $x_i$ (т.е. $x_i$ константа), то достаточно ввести либо ребро $!x_i \rightarrow x_i$, если $x_i = 1$,
либо $x_i \rightarrow !x_i$.

3. Ответ восстанавливается очень просто среди значений $x_0, x_1$ выбирается то, компонента сильной связности
которой стоит позже в порядке топологической сортировки (именно компонента, а не вершина $x_i$).

4. От произвольной таблицы истинности легко перейти к форме $2-CNF$: для этого нужно выбрать все строчки в которых значение
функции равно $0$ и добавить дизъюнкцию в которой переменные равные $1$ взяты с отрицанием, а равные $0$ --- без отрицания.

\textbf{Покраска подпути в дереве на min (offline $O(nlogn)$):}

Для покраски подпути из вершины $x$ в вершину $y$ нужно предварительно подвесить дерево и заменить покраску
$(x, y)$ на $2$ покраски $(x, l)$, $(y, l)$ ($l=lca(x, y)$). Для покраски $(v, p)$ нужно аналогично двоичному подъему разбить запрос по степеням 
и сделать "ленивую" операцию. В конце можно просто "пропушить" все операции в порядке уменьшения степеней двойки.

\textbf{$MVC$ в произвольном графе за $O(\varphi^n)$}

$MVC$ ищется перебором. Предварительно для каждой вершины степени $1$ нужно взять её соседа (если он тоже степени $1$, то
нужно взять только одного из них). Далее в переборе каждый раз ищем вершину с наибольшим количеством непокрытых рёбер.
Теперь нам нужно взять либо её, либо всех её соседей ребро к которым ещё не покрыто.

\textbf{Матричная теорема Кирхгофа:}
Пусть задан неориентированный связный граф с кратными ребрами. Если в матрице смежности графа заменить каждый элемент на противоположный по знаку,
а элемент $a_{ii}$ заменить на степень вершины $i$ (с учетом кратности ребер), то все алгебраические дополнения этой матрицы равны между собой
и равны количеству остовных деревьев этого графа.

\textbf{Матрица Татта:}
Пусть в графе существует совершенное паросочетание, тогда его матрица Татта невырождена. Если алгебраическое дополнение элемента,
соответствующего ребру $(i, j)$, т.е. элемент $A_{j,i}^{-1}$, отличен от нуля, то это ребро может входить в совершенное паросочетание.

