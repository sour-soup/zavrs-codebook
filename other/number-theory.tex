\textbf{Обращение Мебиуса}

$g(m) = \sum\limits_{d | m} f(d) \leftrightarrow f(m) = \sum\limits_{d | m} \mu(d) \cdot g(\frac{m}{d}) $, где $\mu(d) = 
\begin{cases}
0, \text{если d не свободно от квадратов;}\\
(-1)^k, \text{k --- количество простых в разложении d.}\\
\end{cases}$

\textbf{Сумма меньших взаимнопростых:}
$\varphi_1(n) = \frac{n\varphi(n)}{2}=\frac{\varphi(n^2)}{2}$

\textbf{Китайская теорема об остатках (КТО):}

$x \equiv rm_i\ (mod\ m_i)\ (i=\overline{0,n - 1})$

Ответ ищем в виде: $x = x_0 + x_1 m_0 + \ldots + x_{n - 1}m_0m_1\cdot\ldots\cdot m_{n - 2}$

$x_i \leftarrow rm_i,$ for $j < i$ do $x_i \leftarrow (x_i - x_j) m_j^{-1}\ (mod\ m_i)$

Вычисление за линейное время: $x \equiv \sum \limits_{i=0}^{n-1} rm_i \frac{M}{m_i}y_i (mod\ M)$, где $M=\prod \limits_{i=0}^{n-1} m_i$
и $y_i \equiv \left ( \frac{M}{m_i} \right )^{-1} (mod\ m_i)$.

\textbf{Расширенный КТО:}
Заданы числа $n, m, a, b$, нужно найти число $x\equiv a\ (mod\ n), x\equiv b\ (mod\ m)$. Пусть $g=gcd(n, m)$. Если $a\not\equiv b\ (mod\ g)$,
то решения не существует. Пусть $nx_0+my_0=g$, тогда $x=\left(\frac ng x_0 a+\frac mg y_0 b\right)\ mod\ lcm(n, m)$ является наименьшим решением.

\textbf{Решение $x \equiv a^N (mod\ m)$, если $\gcd(a, m) > 1$:}
Пусть $g = gcd(a^N, m)$, найдем наименьшее $k$, что $g | a^k$, так что $a^k=a_1g$, $m=m_1g$.
Тогда $x \equiv a^{N-k}a_1g (mod\ m_1g)$, и, следовательно, $x = x_1g$, где $x_1 \equiv a^{N-k}a_1 (mod\ m_1)$.

\textbf{Сравнение $ax\equiv b\ (mod\ m)$ имеет либо $0$, либо $g = gcd(a, m)$ решений}

\textbf{Функция Кармайкла:}
$\lambda(n)$ --- равна наименьшему показателю $m$, что $a^m \equiv 1 \pmod n, \forall (a, n)=1$.

$\lambda(p^\alpha)=\varphi(p^\alpha), \forall p>2, p\in \mathbb{P}$ или  $p^\alpha\in \{2, 4\}$,

$\lambda(2^\alpha)=\frac{\varphi(2^\alpha)}2, \forall \alpha>2$,

$\lambda(p_1^{\alpha_1} p_2^{\alpha_2}\ldots p^{\alpha_k})=lcm(\lambda(p_1^{\alpha_1}), \lambda(p_2^{\alpha_2}), \ldots, \lambda(p_k^{\alpha_k}))$.

\textbf{Теорема Вильсона:}
Натуральное число $p > 1$ является простым тогда и только тогда, когда $(p - 1)! + 1 \equiv 0\ (mod\ p)$.

\textbf{Критерий Эйлера:}
Пусть $p>2$ - простое число. Число $a$, взаимно простое с $p$, является квадратичным вычетом по модулю $p$ тогда и только тогда, когда
$a^{\frac{p-1}{2}}\equiv 1(mod\ p)$ и является квадратичным невычетом по модулю $p$ тогда и только тогда, когда
$a^{\frac{p-1}{2}}\equiv -1(mod\ p)$.

\textbf{Свойства чисел Фибоначчи:}
$F_{n+1}F_{n-1}-F_{n}^2=(-1)^n$, $F_{n+k}=F_{k}F_{n+1}+F_{k-1}F_{n}$, $F_{2n}=F_{n}( F_{n+1}+F_{n-1})$.

\textbf{Цепные дроби}

$p_n = a_n \cdot p_{n-1} + p_{n-2}$,
$q_n = a_n \cdot q_{n-1} + q_{n-2}$.

\textbf{Цифровой корень:}

Последовательность наименьших целых положительных чисел, которые требуют ровно $n$ итераций извлечения цифрового корня в системе счисления по основанию $b$:
$a_0=0$, $a_1=b$, $a_n=2b^{\frac{a_{n-1}}{b-1}}-1$ при $n > 1$.

\textbf{Числа Гаусса:}
Это комплексные числа, у которых и действительная, и мнимая части целые. Норма числа $a + ib$ определяется как $a^2 + b^2$.
Если $a + ib = (c + id)(e + if)$, то $a^2 + b^2 = (c^2 + d^2)(e^2 + f^2)$. Число $a + ib$ делится на число $c + id$ тогда и только тогда, когда $c^2 + d^2$ делит $ac + bd$ и $bc - ad$.

Гауссово число $a + ib$ является простым тогда и только тогда, когда: 1) либо одно из чисел $a$, $b$ нулевое, а другое - целое простое число вида $\pm (4k + 3)$,
2) либо $a$, $b$ оба отличны от нуля и норма $a^2 + b^2$ - простое натуральное число.

