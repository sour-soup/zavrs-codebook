Дерево, содержащее все неотриц. дроби. На каждом узле стоит медианта $\frac{a + b}{c + d}$, дробей $\frac{a}{b}$ и $\frac{c}{d}$, стоящих в ближайших к этому узлу левом и правом верхних узлах. Все дроби несократимы и появляются ровно $1$ раз.

Нахождение ближайшей дроби к $p / q$ за $O(log^2(p + q))$. Храним текущую левую и правую границы $p_l / q_l$ и $p_r / q_r$.
Бинпоиском ищем макс. $a$: $\frac{p_l + a\,p_r}{q_l + a\,q_r}$ (если идём вправо по дерево), который ничего не ломает, обновляем границы.